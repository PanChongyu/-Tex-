\documentclass[12pt, letterpaper]{article}
\usepackage[utf8]{inputenc}
\usepackage{graphicx}
\usepackage{caption}
\usepackage{bookman}
\usepackage{wrapfig}
\usepackage{geometry}
\usepackage{enumitem}
\usepackage{array}
\usepackage{amsmath}
\usepackage{multirow}
\usepackage{chemfig}

\title{First document}
\author{Pan \thanks{supported by Wang, Pang}}
\date{4/26/2022}
\begin{document}
	
	\begin{titlepage}
		\maketitle
	\end{titlepage}

First example, bold, italics and underline:

Some of the \textbf{greatest} discoveries in \underline{science} were made by \textbf{\emph{accident}}.

\vspace{1.5cm}

Example of italicized text: 

Some of the greatest discoveries in science were made by \emph{accident}.

\vspace{1.5cm}

Example of boldface text:

Some of the \textbf{greatest} discoveries in science were made by accident.

\vspace{1.5cm}

Example of underlined text:

Some of the greatest discoveries in \underline{science} were made by accident.

\vspace{1.5cm}

Example of emphasized text in different contexts:

Some of the greatest \emph{discoveries} in science were made by accident.

\textit{Some of the greatest \emph{discoveries} in science were made by accident.}

\textbf{Some of the greatest \emph{discoveries} in science were made by accident.}

\newpage

\begin{center}
	\begin{tabular}{ |m{5em}| m{2cm} ||m{1cm}| }
		\multicolumn{3}{|c|}{List name} \\
		\hline
		cell1 & cell2 & cell3 \\ 
		\hline
		cell4 & cell5 & cell6 \\  
		cell7 & cell8 & cell9    
	\end{tabular}
\end{center}

\newpage
\begin{align} 
	2x - 5y &=  8 \\ 
	3x + 9y &=  -12 \\
	t = \text{t} \\
	\frac{a}{b}
\end{align}

\newpage
To define chemical formulae you can use units that define the angles

\chemfig{A-[1]B-[7]C}

Absolute angles

\chemfig{A-[:50]B-[:-25]C}

Relative angles

\chemfig{A-[::50]B-[::-25]C}

\end{document}