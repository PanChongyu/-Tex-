\documentclass{article}
\usepackage[UTF8]{ctex}
\usepackage{graphicx}
\usepackage{caption}
\usepackage{bookman}
\usepackage{wrapfig}
\usepackage{geometry}
\usepackage{enumitem}
\geometry{
	a4paper,
	total={170mm,257mm},
	left=20mm,
	top=20mm,
}
\graphicspath{ {./texPic/} }
%Information to be included in the title page:
\title{2022年1季度工作总结}
\author{潘重宇}
\date{2022/4/18}

\begin{document}
\maketitle
\paragraph{在去年11月份从数据分析岗转岗实验室大数据平台运维后,经过一个多月的学习与探索,我基本熟悉了大数据平台的原理、操作与架构,了解了岗位的工作内容与职责。在2022年第一季度的时间里,我主要完成了以下工作内容。}

\begin{itemize}
	\item 4个大数据平台运维工具编写
	\begin{itemize}
		\item HDFS与本地服务器的文件压缩、解压缩互传工具编写。
		\item Hive表格数据文件的自动快照创建与清理工具编写。
		\item HDFS文件本地备份工具编写。
		\item Hive日志文件抓取工具编写。
	\end{itemize}
	\item 2个大数据平台配置运行项目优化
	\begin{itemize}
		\item MySQL配置项优化。
		\item Hive表格数据碎片文件的合并与自动合并配置。
	\end{itemize}
	\item UAT测试集群相关工作
	\begin{itemize}
		\item UAT测试集群的搭建与测试。
		\item UAT测试集群与HDS集群的Kerberos互信建立与测试。
		\item livy/Spark3组件在HUE页面集成。
	\end{itemize}
	\item 2项集群bug修复
	\begin{itemize}
		\item 集群log4j2组件远程代码注入漏洞修复。
		\item HDFS文件系统高权限问题的研究与解决。
	\end{itemize}
	\item 大数据平台日常运维项目
	\begin{itemize}
		\item 天网、资金清算与数据提取项目组30多张数据表的导入加载。
		\item 每日加载错误表格手动修复。
	\end{itemize}
\end{itemize}

\paragraph{通过实际项目工作的的进行,我更好的学习与掌握了大数据平台的相关知识,尤其是HDFS系统的运行原理和部分API的使用方法,并通过这编写了自动化工具,解决了一些运维工作严重依赖于手动操作的问题,提高了自动化水平。同时对大数据平台的运行做了些许优化,能够提高平台用户的数据查询、保存的效率。}
\end{document}