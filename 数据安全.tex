\documentclass{beamer}
\usepackage[UTF8]{ctex}
\usepackage{graphicx}
\usepackage{caption}
\usepackage{bookman}
\usetheme{Madrid}
\usepackage{wrapfig}
\graphicspath{ {./texPic/} }
%Information to be included in the title page:
\title{数据安全}
\author{潘重宇}
\date{2022/8/8}

\begin{document}
	
	\frame{\titlepage}
	
	\AtBeginSection[]
	{
		\begin{frame}
			\frametitle{目录}
			\tableofcontents[currentsection]
		\end{frame}
	}
	
	\section{数据安全概述}
	\begin{frame}
		\frametitle{数据安全概述}
		\begin{block}{数据是什么}
			数据,是指任何以电子或者其他方式对信息的记录。
		\end{block}
	\begin{block}{数据处理}
		数据的收集、存储、使用、加工、传输、提供、公开等过程。
	\end{block}
	\begin{block}{数据安全}
		通过采取必要措施,确保数据处于有效保护和合法利用的状态,以及具备保障持续安全状态的能力。
	\end{block}
	\quad2021年9月1日,《中华人民共和国数据安全法》正式施行,此项立法进一步确保了数据处于有效保护和合法利用的状态,以更好保护个人和组织的合法权益,维护国家主权、安全和发展利益。
	\end{frame}
	
	\begin{frame}
		\frametitle{数据安全的重要性}
		\quad
		网络空间的安全不仅包括网络本身的安全,而且包括数据、信息系统、智能系统、信息物理融合系统等多个方面的广义安全。数据安全是网络空间安全的基础,是国家安全的重要组成部分。数据安全与网络安全、信息安全、系统安全、内容安全和信息物理融合系统安全有着密不可分的关系。
		\\
		\quad
		数据安全与网络安全密切相关,是国家主权、国家安全的重要组成部分。
	\end{frame}
	
	\section{数据安全方案分析}
	\begin{frame}
		\frametitle{基于生命周期的数据安全方案}
		\begin{itemize}
			\item<1-> 数据采集安全。
			\item<2-> 数据传输安全。
			\item<3-> 数据存储安全。
			\item<4-> 数据处理安全。
			\item<5-> 数据交换安全。
			\item<6-> 数据销毁安全。
		\end{itemize}
	\end{frame}
	
	\begin{frame}
		\frametitle{数据采集安全}
		\begin{block}{数据采集阶段}
			数据采集阶段是组织内部系统中新产生数据,以及从外部系统收集数据的阶段。如何保证海量数据的安全可治理,是这一阶段的重点。
		\end{block}
		\quad
		需要保护什么数据、这些数据分布在哪里、风险程度如何,对大部分人来说都是企业数据安全管理的盲区。有效掌握所需保护数据的分布,是做好数据安全工作的第一步。
	\end{frame}
	
	\begin{frame}
		\frametitle{数据传输安全}
		\begin{block}{数据传输阶段}
			数据传输阶段是数据从一个实体传输到另一个实体的阶段。
		\end{block}
		\quad
		在数据抽取、数据导入、数据共享、数据发布、数据展示和数据下载的过程中,要做好鉴权、加密的工作,防止中间人泄露、篡改数据。
	\end{frame}
	
	\begin{frame}
	\frametitle{数据存储安全}
	\begin{block}{数据存储阶段}
		数据以任何数字格式进行存储的阶段。
	\end{block}
	\quad
	传统介质上做好文件加密和磁盘加密;在数据库、数据中台等结构化存储上做好脱敏、字段加密;在对象类、日志类、缓存类的非结构存储上做好脱敏、客户端加密、服务端加密等安全措施。
	\end{frame}

	\begin{frame}
	\frametitle{数据处理安全}
	\quad
	对于需要重点处理的数据实现脱敏化处理,确保使用者无从泄露重要数据。
	\end{frame}

	\begin{frame}
	\frametitle{监控审计}
	\quad
	针对数据生存周期各阶段开展安全监控和审计,以保证对数据的访问和操作均得到有效的监控和审计。基于组织对终端设备层面的数据保护要求,针对组织内部的工作终端采取相应的技术和管理方案,例如采用堡垒机方案。
	\end{frame}
	
\end{document}